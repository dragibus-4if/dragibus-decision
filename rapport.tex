\documentclass[a4paper, 11pt]{article}

\usepackage[utf8]{inputenc}
\usepackage[frenchb]{babel}
\usepackage[T1]{fontenc}
\usepackage{textcomp}
\usepackage{amsmath,amssymb}
\usepackage{lmodern}
\usepackage[a4paper]{geometry}
\usepackage{graphicx}
\usepackage{xcolor}
\usepackage{microtype}
\usepackage{listings}
\usepackage{hyperref}

\title{Aide à la décision}
\author{Dragibus}
\date{}

\begin{document}
\maketitle
\tableofcontents
\newpage

\section{Introduction}
Cette étude se déroulera en trois parties. Dans la première nous considèrerons
le nombre de produits de chaque type à produire pour optimiser les critères
retens par chaque cadre de l'entreprise ADécision. La deuxième partie
comprendra une analyse de tous les critères considérés dans la première partie
et la recherche d'une solution de compromis satisfaisant chacun d'entre-eux. La
dernière partie consistera en la sélection d'une solution finale parmis celles
proposées précédemment.

\subsection{Définition}
Nous définissons ici les notations que nous utiliserons dans la suite de ce
rapport et exprimmons certaines notions sous forme de fonctions mathématiques.
\\
Soit $E_M = \left\{1, 2, 3, 4, 5, 6, 7\right\} $ l'ensemble des machines. \\
Soit $E_{MP} = \left\{MP1, MP2, MP3\right\} $ l'ensemble des matières premières. \\
Soit $E_P = \left\{A, B, C, D, E, F\right\} $ l'ensemble des produits. \\
Soit $B$ le bénéfice de l'entreprise. \\
Soit $TTH$ le temps de travail hebdomadaire. \\
Soit $X(i)$ le nombre de produit $i\in E_P$ fait. \\
Soit $X = \left\{X(1), X(2), X(3), X(4), X(5), X(6), X(7)\right\}$ le vecteur
définissant les quantités de chaque produit à fabriquer.\\
Soit $S(p)$ la quantité en stock de la matière première $p\in E_{MP}$. \\
Soit $TM(m)$ le temps d'utilisation de la machine $m\in E_M$.\\
Soit $G(i)$ le gain engendré par le produit $i$. \\
Soit $V(i)$ le prix de vente du produit $i$. \\
Soit $P(i)$ la perte engendrée par le produit $i$. \\
Soit $Q(i, p)$ la quantité de matière $p$ nécessaire pour le produit $i$. \\
Soit $A(p)$ le prix d'achat de la matière $p$. \\
Soit $C(m)$ le cout de la machine $m$. \\
Soit $T(i, m)$ le temps d'usinage de la machine $m$ pour le produit $i$. \\

$$
\begin{array}{r l}
    G(i) =  & V(i)\cdot X(i) \\
    P(i) =  & X(i)\left(\sum_{p\in E_{MP}} A(p)Q(i, p)+ \sum_{m\in E_M} C(m)T(i, m)\right) \\
    B =     & \sum_{i\in E_P} G(i) - P(i) \\
    TM(m) = & \sum_{i\in E_P} X(i)T(i, m) \\
    TTH =   & \sum_{m\in E_M} TM(m) \\
    S(p) =  & \sum_{i\in E_P} X(i)Q(i, p) \\
\end{array}
$$

\subsection{Ensemble des contraintes}
Cette section exprime sous forme mathématique les différentes contraintes
inhérentes a fonctionnement de l'entreprise.

$$
\begin{array}{r l}
    \forall i\in E_P, & X(i) > 0 \\
    \forall m\in E_M, & TM(m) \in [0, 4800\mbox{min}] \\
                      & TTH \in [0, 4800\mbox{min}] \\
                      & B > 0 \\
                      & S(MP1) \in [0, 650] \\
                      & S(MP2) \in [0, 820] \\
                      & S(MP3) \in [0, 585] \\
\end{array}
$$

\section{Optimisation monocritère}
Dans cette partie, nous considérons un par un les différents critères retenus
par les différents cadres de l’entreprise ADécision. Nous commençons par
exprimer le problème sous forme d’un modèle mathématique. Il est possible de
modéliser chacune des situations suivantes sous forme d’une fonction f de X à
minimiser en respectant la condition $A . X \leq b$;

$A = \begin{pmatrix}
11&15&0&5&0&10 \\
0&1&2&8&7&12\\
12&1&11&0&10&15\\
2&10&5&4&13&7\\
15&0&0&0&10&25\\
5&5&13&12&8&0\\
5&3&5&28&0&7\\
1&1&1&5&0&2\\
2&2&1&0&2&1\\
1&0&3&2&6&0
\end{pmatrix}$\\

$b = \begin{pmatrix}
4800\\
4800\\
4800\\
4800\\
4800\\
4800\\
4800\\
650\\
820\\
585\\
\end{pmatrix}$


\subsection{Comptable}
L'objectif du comptable est de maximiser les bénéfices de l'entreprise.
Il suffit donc de maximiser la fonction de bénéfice définie précédemment.

$G(X) = X\cdot(28~20~30~37~45~22)^t$ \\

$$
\begin{array}{r l}
    \mbox{CoutAchat}(A) = & (1~2~1)\cdot(3~4~2)^t = 13 \\
    \mbox{CoutAchat}(B) = & (1~2~0)\cdot(3~4~2)^t = 11 \\
    \mbox{CoutAchat}(C) = & (1~1~3)\cdot(3~4~2)^t = 13 \\
    \mbox{CoutAchat}(D) = & (5~0~2)\cdot(3~4~2)^t = 19 \\
    \mbox{CoutAchat}(E) = & (0~2~6)\cdot(3~4~2)^t = 20 \\
    \mbox{CoutAchat}(F) = & (2~1~0)\cdot(3~4~2)^t = 10 \\
\end{array}
$$

$$
\begin{array}{r r l l}
    \mbox{CoupMachine}(A) = & (11~0~12~2~15~5~5)\cdot  & \frac{1}{60}(2~2~1~1~2~3~1)^t = & \frac{86}{60} \\
    \mbox{CoupMachine}(B) = & (15~1~1~10~0~5~3)\cdot   & \frac{1}{60}(2~2~1~1~2~3~1)^t = & \frac{61}{60} \\
    \mbox{CoupMachine}(C) = & (0~2~11~5~0~13~5)\cdot   & \frac{1}{60}(2~2~1~1~2~3~1)^t = & \frac{64}{60} \\
    \mbox{CoupMachine}(D) = & (5~8~0~4~0~12~28)\cdot   & \frac{1}{60}(2~2~1~1~2~3~1)^t = & \frac{94}{60} \\
    \mbox{CoupMachine}(E) = & (0~7~10~13~10~8~0)\cdot  & \frac{1}{60}(2~2~1~1~2~3~1)^t = & \frac{71}{60} \\
    \mbox{CoupMachine}(D) = & (10~12~15~7~25~0~7)\cdot & \frac{1}{60}(2~2~1~1~2~3~1)^t = & \frac{123}{60} \\
\end{array}
$$

$$
\begin{array}{r l}
    P(X) = & X\cdot\left((13~11~13~19~20~10) + \frac{1}{60}(86~61~64~94~71~123)\right)^t \\
         = & X\cdot\frac{1}{60}(866~721~844~1234~1271~723)^t \\
\end{array}
$$

$$
\begin{array}{r l}
    B(X) = & G(X) - P(X) \\
         = & X\cdot\frac{1}{60}(814~479~956~986~1429~597)^t \\
         = & X\cdot(13.5667~7.98333~15.9333~16.4333~23.8167~9.95)^t \\
\end{array}
$$

L'on détermine donc le vecter X maximisant les bénéfices selon les contraintes du problème en minimisant la fonction :\\
$$
\begin{array}{rl}
    f(X) = & -B(X) \\
         = & X\cdot(-13.5667~-7.98333~-15.9333~-16.4333~-23.8167~-9.95)^t
\end{array}
$$

sous les contraintes
$$
\left\{
    \begin{split}
        A\cdot X \leq b\\
        0 \leq X
    \end{split}
\right.
$$

L'on obtient le résultat suivant : \\
$ X = (235.6250~98.3929~101.3393~22.6786~0~50.6280) $ \\

Cela correspond à un bénéfice de 6473.2 €.

\subsection{Responsable d'atelier}
L'objectif du responsable atelier est de maximiser le nombre de produits fabriqués.
Il faut donc minimiser la fonction suivante : \\
$$
\begin{array}{rl}
    f(X) = & -X(A) + -X(B) + -X(C) + -X(D) + -X(E) + -X(F) \\
         = & X\cdot(-1~-1~-1~-1~-1~-1)^t
\end{array}
$$

en respectant les contraintes suivantes : \\
$$
  \left\{
    \begin{split}
     A\cdot X \leq b\\ 
     0 \leq X
    \end{split}
  \right.
$$

L'on obtient le résultat suivant : $ X = (0~252.5~195.0~0~0~101.25) $ \\

\subsection{Responsable commercial}
L'objectif du responsable commercial est d'équilibrer les quantités faites par famille de produits.
La quantité de production des produits A, B et C doit être la même que celle des produits D, E et F : \\
$X(A) + X(B) + X(C) = X(D) + X(E) + X(F)$ \\

La fonction a minimiser est donc celle représentant la différence de production des deux familles : \\
Si l’on souhaite maximiser la production tout en respectant la contrainte que nous
nous sommes fixés précédemment, on réutilise la fonction $f$ utilisée par le
responsable d’atelier : $ f(X) = X\cdot(1~1~1~1~1~1)^t $

On utilise alors le meme procédé qu'avec le responsabme d'atelier en ajoutant
la contrainte précédemment trouvé représenté par l'équation : \\
$ f_c(X) = 0, f_c(X) = X\cdot(1~1~1~-1~-1~-1)^t $.

On trouve alors comme solution $X = (0~206.03~43.26~14.70~70.97~163.6) $ \\

Si l’on souhaite maximiser les bénéfices tout en respectant l’objectif que nous
nous sommes fixés précédemment, on réutilise la fonction $f$ utilisée par le
comptable. On trouve alors le meme $X$ qu'avant. \\

On peut donc voir que, en maximisant le bénéfice, ou en maximisant le nombre de
pièces produites, le fait d’égaliser le nombre de produits par famille fait
converger les deux résultats. On trouve un bénéfice de 5893,8 unités d’argent,
ce qui correspond à 91,05\% du bénéfice maximal.


\subsection{Responsable des stocks}
Le responsable des stocks veut minimiser la taille des stocks, c'est à dire le
nombre de produits ainsi que la quantité de matières premières stockés. L'on
cherche aussi à garder un bénéfice positif permettant à l'entreprise de
continuer sa croissance. On considère alors la taille de stock initial au début
de la semaine de production plus l'espace que prendront les produits fabriqués.

La fonction a minimiser est donc la suivante : \\
$$
\begin{array}{rl}
    f(X) = & 2\cdot \sum_{p\in E_{MP}} S(p) \\
         = & 5 X(A) + 6 X(B) + 8 X(D) + 9 X(E) + 4 X(F) \\
         = & X\cdot(5~6~8~9~4)^t
\end{array}
$$

Le stock ne peut être négatif donc on contraint a être supérieur à 0. Le
minimum de cette fonction est alors 0. Or l'entreprise doit toujours pouvoir
produire.Il est donc nécessaire de rajouter des contraintes afin d'avoir un
résultats réalistes. On contraint le système en prenant alors d'autres
critères. On peut maximiser le bénéfice ou la production par exemple. Nous
avons choisis de traduire l’activité de l’entreprise grâce a son bénéfice. \\

% TODO rajouter l'image qu'il y a sur le drive
% + donner une explication (donc changer un peu le paragraphe suivant
\textbf{TODO} \\

La solution est considérée comme viable si le benefice est au moins supérieur à
70\% du bénéfice total. On remarque un point d’inflection pour un stock de
407.2 unité correspondant a un bénéfice de 5742 unité d’argent (soit 88.7\% du
bénéfice maximum). Il n’est donc pas judicieux de chercher une solution avec
un bénéfice plus élévé car cela entrainerais une augmentation beaucoup plus
rapide des stocks. \\

On trouve alors : $ X = (319.5453 ~0.3081 ~87.1207 ~0.0000 ~0.6821 ~0.0000) $

\subsection{Responsable du personnel}
Le responsable du personnel souhaite limiter l'usage des machines 1 et 5.
% TODO
\\ \textbf{TODO} \\

\section{Programmation linéaire multicritère}

\section{Analyse multicritère}

\end{document}
